\documentclass[a5paper,11pt,english]{article}
\usepackage{.styles/basic}
\usepackage{.styles/envs}

%%%%%%% Title %%%%%%%%%%%%%%%%%%%%%%%%%%%
\title{\textbf{Algebraic Geometry} - Exercise Sheet 5}
\author{Tor Gjone \& Paul}

%%%%%%% Definitions %%%%%%%%%%%%%%%%%%%%%


% Algebraic Geometry stuff
\def\A{\mathbb{A}}
\DeclareMathOperator{\rad}{rad}
\DeclareMathOperator{\tr}{tr}

\DeclareMathOperator{\Sh}{Sh}
\DeclareMathOperator{\op}{^{op}}
\DeclareMathOperator{\Id}{Id}
\DeclareMathOperator{\Ouv}{Ouv}

\DeclareMathOperator{\Ann}{Ann}
\DeclareMathOperator{\Supp}{Supp}
\DeclareMathOperator{\res}{res}
\newcommand{\B}{\mathcal{B}}
\newcommand{\F}{\mathcal{F}}
\newcommand{\G}{\mathcal{G}}


% Direct limit and colimit
\setlength{\parindent}{0cm}
\setlength{\parskip}{3pt}

\makeatletter
\newcommand{\lim@}[2]{%
  \vtop{\m@th\ialign{##\cr
    \hfil$#1\operator@font lim$\hfil\cr
    \noalign{\nointerlineskip\kern1.5\ex@}#2\cr
    \noalign{\nointerlineskip\kern-\ex@}\cr}}%
}
\newcommand{\colim}{%
  \mathop{\mathpalette\varlim@{\rightarrowfill@\scriptscriptstyle}}\nmlimits@
}
\newcommand{\invlim}{%
  \mathop{\mathpalette\varlim@{\leftarrowfill@\scriptscriptstyle}}\nmlimits@
}
\makeatother

% Tor's stuff
\renewcommand{\i}{^{-1}}

\newcommand{\til}{\tilde}


\newcommand{\sm}{\setminus}
\newcommand{\es}{\emptyset}
\newcommand{\se}{\subseteq}

\newcommand{\oc}[1]{\overset{\circ}{#1}}
\newcommand{\into}{\hookrightarrow}
\newcommand{\nb}{\nabla}
\renewcommand{\hat}{\widehat}

\usepackage{scalerel,stackengine}
\stackMath
\renewcommand\widehat[1]{%
\savestack{\tmpbox}{\stretchto{%
  \scaleto{%
    \scalerel*[\widthof{\ensuremath{#1}}]{\kern.1pt\mathchar"0362\kern.1pt}%
    {\rule{0ex}{\textheight}}%WIDTH-LIMITED CIRCUMFLEX
  }{\textheight}% 
}{2.4ex}}%
\stackon[-6.9pt]{#1}{\tmpbox}%
}




\newcommand{\diagSquare}[8]{
\begin{tikzpicture}[node distance=2cm, auto]
\node (a)              { $ #5 $ };
\node (b) [right of=a] { $ #6 $ };
\node (c) [below of=a] { $ #7 $ };
\node (d) [right of=c] { $ #8 $ };
\draw[-to] (a) to node { $ #1 $ } (b);
\draw[-to] (a) to node { $ #2 $ } (c);
\draw[-to] (b) to node { $ #3 $ } (d);
\draw[-to] (c) to node { $ #4 $ } (d);
\end{tikzpicture}
}


\newcommand{\congto}{\xrightarrow{\cong}}
\newcommand{\incto}[1][]{ \xhookrightarrow{#1} }
\newcommand{\xto}[1]{\xrightarrow{#1}}


% Projective spaces
\def\RP{\mathbb{RP}}
\def\CP{\mathbb{CP}}
\def\HP{\mathbb{HP}}


% Commen Cathegories
\DeclareMathOperator{\Top}{Top}
\DeclareMathOperator{\Ab}{Ab}
\DeclareMathOperator{\Cat}{Cat}
\DeclareMathOperator{\CAT}{CAT}
\DeclareMathOperator{\Mod}{Mod}
\DeclareMathOperator{\Ring}{Ring}
\DeclareMathOperator{\Group}{Group}
\DeclareMathOperator{\Sets}{Sets}


% Homological Algebra
\DeclareMathOperator{\Hom}{Hom}
\DeclareMathOperator{\Tor}{Tor}
\DeclareMathOperator{\Ext}{Ext}


% Common Lie Groups 
\DeclareMathOperator{\GL}{GL}
\DeclareMathOperator{\SU}{SU}
\DeclareMathOperator{\U}{U}
\DeclareMathOperator{\Sp}{Sp}

% Maths Operators
\DeclareMathOperator{\id}{id}
\DeclareMathOperator{\pr}{pr}


\DeclareMathOperator{\Ker}{Ker}
\DeclareMathOperator{\im}{Im}


% Standard setts.
\def \N {\mathbb{N}}
\def \Z {\mathbb{Z}}
\def \Q {\mathbb{Q}}
\def \R {\mathbb{R}}
\def \C {\mathbb{C}}
\def \H {\mathbb{H}}

\def \E {\mathbb{E}}
\def \Z {\mathbb{Z}}
\def \I {\mathbb{I}}
\def \J {\mathbb{J}}

% Vector calculus.
\newcommand{\dif}[3][]{
	\ensuremath{\frac{d^{#1} {#2}}{d {#3}^{#1}}}}
\newcommand{\pdif}[3][]{
	\ensuremath{\frac{\partial^{#1} {#2}}{\partial {#3}^{#1}}}}

% Vectors and matricies.
\newcommand{\mat}[1]{\begin{matrix} #1 \end{matrix}}
\newcommand{\pmat}[1]{\begin{pmatrix} #1 \end{pmatrix}}
\newcommand{\bmat}[1]{\begin{bmatrix} #1 \end{bmatrix}}

% Add space around the argument.
\newcommand{\qq}[1]{\quad#1\quad}
\newcommand{\q}[1]{\:\:#1\:\:}

% Implications
\newcommand{\la}{\ensuremath{\Longleftarrow}}
\newcommand{\ra}{\ensuremath{\Longrightarrow}}
\newcommand{\lra}{\ensuremath{\Longleftrightarrow}}

\newcommand{\pwf}[1]{\begin{cases} #1 \end{cases}}
\newcommand{\tif}{\text{if}}
\newcommand{\tand}{\text{and}}

% Shorthand
\newcommand{\vphi}{\varphi}
\newcommand{\veps}{\varepsilon}

\newcommand{\<}[1]{\langle #1 \rangle}

% Notation
\newcommand{\wddef}[1]{\underline{#1}}
\newcommand{\pref}[1]{(\ref{#1})}


% Maths Operators
\theoremstyle{plain}
\theoremstyle{definition}
\newtheorem{thrm}{Theorem}[section]
\newtheorem{prop}[thrm]{Proposition}
\newtheorem{corol}[thrm]{Corollary}
\newtheorem{lemma}[thrm]{Lemma}

\newtheorem{defn}[thrm]{Definition}
\newtheorem{exmp}[thrm]{Example}
\newtheorem{clame}[thrm]{Clame}


\theoremstyle{remark}
% \newtheorem{remark}[thrm]{\normalfont\large\textit Remark}
\newtheorem{remark}[thrm]{Remark}
\newtheorem{note}[thrm]{Note}


\newSimpleHeaderEnvironment{exercise}{Exercise }

\DeclareMathOperator{\Fun}{Fun}
\DeclareMathOperator{\Spec}{Spec}

\renewcommand{\O}{\mathcal{O}}
\newcommand{\Cc}{\mathcal{C}}
\newcommand{\Dd}{\mathcal{D}}

\newsavebox{\pullback}
\sbox\pullback{%
\begin{tikzpicture}%
\draw (0,0) -- (1ex,0ex);%
\draw (1ex,0ex) -- (1ex,1ex);%
\end{tikzpicture}}


%%%%%%%% Content %%%%%%%%%%%%%%%%%%%%%%%%
\begin{document}
\mmaketitle


\begin{exercise}[2]
\begin{enumerate}
\item % 1
We'll denote the map by $\Psi$. 
\[ \Psi : \Hom(h_X, F) \to F(X); \q \eta \mapsto \eta_X(\Id_X)). \]

Let $\eta \in \Hom(h_X, F)$, ie. a natural transformation from $h_X$ to $F$. 
Then for $Y,Z \in \Cc$ and $f: Y \to Z$, the following diagram commutes
\[ \begin{tikzcd}
\Hom_\Cc(Z,X) \arrow[r,"\eta_Y"] \arrow[d, "h_X f"] & \F(Y) \arrow[d, "F(f)"] \\
\Hom_\Cc(Y,X) \arrow[r,"\eta_Z"] & F(Y)
\end{tikzcd} \]
In particular, the diagram commutes for $f : Y \to X$. So
\begin{equation}
\eta_Y(f) = \eta_Y(f\circ \Id_X) = F(f)(\eta_X(\Id_X)). 
\end{equation}
So if $\xi \in \Hom(h_X, F)$, such that $\eta_x(\Id_X) = \xi_X(\Id_X)$, ie.
$\Psi(\eta) = \Psi(\xi)$. Then $\eta_Y(f) = \xi_Y(f)$, for all $f: Y \to X$, so
$\eta = \xi$. So $\Psi$ is injective. 

Furthermore, if $x\in F(X)$, we may define $\eta^x \in \Hom(\eta_X,F)$, by 
\[ \eta^x_Y(f) := F(f)(x), \qquad \forall Y \in \Cc, \: f:Y \to X. \]
Then 
\[ \Psi(\eta^x) = \eta^x(\Id_X) = F(\Id_X)(x) = \Id_X(x) = x. \] 
So $\Psi$ is also surjective. It remains to show that $\eta^x$ satisfies the naturally
conditions (ie. the commutative diagram above). Let $Y,Z \in \Cc$, $g : Y \to
Z$, and $f: Y \to X$. Then
\begin{align*}
(\eta^x_Z \circ h_X g)(f) &= \eta^x_Z (f \circ g)  = F(f \circ g)(x)  =
F(g)(F(f)(x)) = F(g)(\eta^x_Y(f)).
\end{align*}


Consider the case $F = h_Y$, then the Yoneda lemma implies that  
\[ \Hom(h_X,h_Y) \cong \Hom(X,Y). \]
So the functor $\Cc \mapsto \Fun(\C\op, \Sets); X \mapsto h_X$ is fully
faithful.

\item % 2
% A morphism $f: X \to Y$ of schemes over $S$ is given by a mo
The first bijection follows immediately from part 1. So what we need to show is
the second bijections.

Let $\Xi: \Hom_S(X,Y) \to \Hom({h_X}_{|\Dd},{h_Y}_{|\Dd})$ be the composition of
the Yoneda map $\Psi$ from part 1, and the restriction sub-category $\Dd$. More
explicitly $Xi$ is the composition
\[ \begin{tikzcd}[row sep=.6em]
\Hom_S(X,Y) \arrow[r]& \Hom(h_X,x_Y) \arrow[r] & \Hom({h_X}_{|\Dd},
{h_Y}_{|\Dd}) \\
f \arrow[r, mapsto]& \eta^f \arrow[r, mapsto]& \eta^f_{|\Dd}
\end{tikzcd} \]
where $\eta^f$ is defined (like in part 1, with $F = h_Y$) by 
\[ \eta^f_Z(g) = h_Y(g)(f) = f\circ g, \qquad \forall Z \in \Cc, \quad g:Z \to
X, \]
and $\eta^f_{|\Dd}$ is the restriction of a natural transformation to a natural
transformation between the restriction of the functors.

To show the second bijection we will construct a map $\Phi: \Hom({h_X}_{|\Dd},
{h_Y}_{|\Dd}) \to \Hom(X,Y)$ inverse to $\Xi$.

Let $\eta \in \Hom({h_X}_{\Dd}, {h_Y}_{\Dd})$ and $X = \bigcup_{i\in I}$ such that
$X_i$ is an affine scheme. Let 
\[ \eta_i := \eta_{X_i} : \Hom_S(X_i, X) \to \Hom_S(X_i, Y) \]
and $\iota_{i,j} : X_i \cap X_j \into X_i$, with $(\iota_{i,j})^\#_x = \id :
\O_{X_i,x} \to \O_{X_i\cap X_j, x} = \O_{X_i,x}$, for all $x\in X_i \cap X_j$. Then, by the neutrality of $\eta$, 
the diagram 
\[ \begin{tikzcd}
\Hom_S(X_i, X) \arrow[r, "\eta_i"] \arrow[d,"h_X \iota_{i,j}"] &
\Hom_S(X_i, Y) \arrow[d, "h_Y\iota_{i,j}"] \\
\Hom_S(X_i\cap X_j, X) \arrow[r, "\eta_{X_i\cap X_j}"] &
\Hom_S(X_i\cap X_j, Y) 
\end{tikzcd} \]
commutes. Equivalently, for all morphisms $f : X_i \to X$, then 
\begin{equation}
\eta_i(f) \circ \iota_{i,j} = \eta_{X_i \cap X_j} (f \circ \iota{i,j})
\qq\tand
(\iota_i)^\# \circ \eta_i(f^\#) = \eta_{X_i \cap X_j} \left( (\iota_i)^\# \circ
f^\#  \right)
\label{eq:2.2}
\end{equation}

Let $\iota_i : X_i \into X$, be the inclusion, with $(\iota_i)^\#_x = \id : \O_{X,x}
\to \O_{X_i,x} = \O_{X,x}$. 
Define $\Psi(\eta): X \to Y$, for all $x \in X$, by
\[ \Psi(\eta)(x) := \eta_i(\iota_i)(x) \qq\tand \Psi(\eta)_x^\# :=
\eta_i\left((\iota_i)_x^\#\right), \qquad\tif \quad x \in X_i. \]
Clearly $\iota_i$ is a morphism of schemes over $S$, so, if well defined,
$\Psi(\eta)$ is also a morphism of schemes over $S$.

It follows from equation \pref{eq:2.2} that this is well-defined. Suppose $x \in
X_i \cap X_j$, then
\[ \eta_i(\iota_i)(x) = (\eta_i(\iota_i) \circ \iota_{i,j})(x) 
= \eta_{X_i\cap X_j} (\iota_i \circ \iota_{i,j})(x) 
= \eta_{X_i\cap X_j} (\iota_j \circ \iota_{j,i})(x)
= \eta_j(\iota_j)(x), \]
and 
\begin{align*}
\eta_i\left((\iota_i)_x^\#\right)
= (\iota_j)_x^\# \circ \eta_i\left( (\iota_i)_x^\#   \right)
= \eta_{X_i \cap X_j} \left( (\iota_i)_x^\# \circ (\iota_j)_x^\#  \right)
= \eta_j\left((\iota_j)_x^\#\right).
\end{align*}
The first and third equality follows from $(\iota_j)_x^\# = \id$, and the second
by equation \pref{eq:2.2}. 

It remains to show that $\Psi$ defines an inverse of $\Xi$. Let $f : X \to Y$ be
a morphism of schemes over $S$. Then for $x \in X_i$ 
\begin{align*}
\Psi(\Xi(f))(x) &= \eta^f_{X_i}(\iota_i)(x) = (f \circ \iota_i)(x) = f(x), \quad
\tand \\
\Psi(\Xi(f))_x^\# &= \eta^f_{X_i} ( (\iota_i)_x^\# ) = (\iota_i)_x^\# \circ
f_x^\# = f_x^\#.
\end{align*}
Conversely, let $\eta \in \Hom({h_X}_{|\Dd}, {h_Y}_{|\Dd})$, $g: Z \to X$, where
$Z$ affine scheme, $z \in Z$ and $i \in I$, such that $g(z) \in X_i$. Then
\begin{align*}
\Xi(\Psi(\eta))(g)(z) &= \eta_Z^{\Psi(\eta)} (g)(z) \\
&= \Psi(\eta) (g(z)) \\
&= \eta_i(\iota_i)(g(z))
\end{align*}
% \todo{Argue $\eta_i(\iota_i)(g(z)) = \eta_Z(g)(z)$.}
\end{enumerate}


\end{exercise}

\newpage
\begin{exercise}[3]
We have a pull-back diagram, for each of the two fibre-products.

\renewcommand{\|}{\hspace{-.25em}\mid\hspace{-.25em}}

\[ \begin{tikzcd}
X\times_S Y \arrow[r,"p_X"] \arrow[d, "p_Y"] 
\arrow[dr, phantom, "\usebox\pullback" , very near start, color=black]
& X \arrow[d, "f"] \\
Y \arrow[r, "g"] & S
\end{tikzcd}
\qquad
\begin{tikzcd}
\|X\| \times_{|S|} \|Y\| \arrow[r,"q_X"] \arrow[d, "q_Y"] 
\arrow[dr, phantom, "\usebox\pullback" , very near start, color=black]
& \|X\| \arrow[d, "|f|"] \\
\|Y\| \arrow[r,"|g|"] & \|S\|
\end{tikzcd} \]

By the first diagram $|p_X|: |X\times_S Y| \to |X|$ and $|p_Y|: |X\times_S Y|
\to |Y|$, such that $|f|\circ |p_X| = |g|\circ |p_Y|$. So by the universal prop
of the second there is a unique map $\pi : |X \times_S Y| \to |X|\times_{|S|}
|Y|$. 

To show that $\pi$ is surjective. Consider $x \in X$, $y\in Y$, $T = \Spec(k)$,
for some filed $k$ and define 
\[ \xi_x : T \to X; \star \to x, \qq\tand 
\zeta_y : T \to Y; \star \to y, \]
with $\xi_x^\# = 0: \O_{X,x} \to k$ and $\zeta_y^\# = 0 : \O_{Y,y} \to k$.

Then if $f(x)=g(y)$, $(T,\xi_x, \zeta_y)$, defines a cone over the first
pull-back diagram
and $(|T|,|\xi_x|, |\zeta_y|)$ defines a cone over the second pull-back diagram.
Let $\phi_{x,y}$ and $\psi_{x,y}$ denote the unique maps determined by the
universal properties of the first and second pull-backs, respectively. 
By the uniqueness of $\psi_{x,y}$, $\psi_{x,y} = \phi_{x,y} \circ \pi$ and
$\psi_{x,y}(\star) = [(x,y)] \in |X|\times_{|S|} |Y|$, so $[(x,y)] \in \im\pi$.
Since $x,y$ was arbitrary, such that $f(x)=g(y)$, $\pi$ is surjective.


% The fibre over $[(x,y)]$ is dete

\end{exercise}




\end{document}
