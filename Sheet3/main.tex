\documentclass[a4paper,11pt,english]{article}
\usepackage{.styles/basic}
\usepackage{.styles/envs}

\usepackage{tikz}
\usepackage{tikz-cd}

%%%%%%% Title %%%%%%%%%%%%%%%%%%%%%%%%%%%
\title{\textbf{Algebraic Geometry} - Exercise Sheet 3}
\author{Tor Gjone \& Paul}

%%%%%%% Definitions %%%%%%%%%%%%%%%%%%%%%


\newsavebox{\pullback}
\sbox\pullback{%
\begin{tikzpicture}%
\draw (0,0) -- (1ex,0ex) -- (1ex, 1ex);
\end{tikzpicture}}


% Algebraic Geometry stuff
\def\A{\mathbb{A}}
\DeclareMathOperator{\rad}{rad}
\DeclareMathOperator{\tr}{tr}

\DeclareMathOperator{\Sh}{Sh}
\DeclareMathOperator{\op}{^{op}}
\DeclareMathOperator{\Id}{Id}
\DeclareMathOperator{\Ouv}{Ouv}

\DeclareMathOperator{\Ann}{Ann}
\DeclareMathOperator{\Supp}{Supp}
\DeclareMathOperator{\res}{res}
\newcommand{\B}{\mathcal{B}}
\newcommand{\F}{\mathcal{F}}
\newcommand{\G}{\mathcal{G}}


% Direct limit and colimit
\setlength{\parindent}{0cm}
\setlength{\parskip}{3pt}

\makeatletter
\newcommand{\lim@}[2]{%
  \vtop{\m@th\ialign{##\cr
    \hfil$#1\operator@font lim$\hfil\cr
    \noalign{\nointerlineskip\kern1.5\ex@}#2\cr
    \noalign{\nointerlineskip\kern-\ex@}\cr}}%
}
\newcommand{\colim}{%
  \mathop{\mathpalette\varlim@{\rightarrowfill@\scriptscriptstyle}}\nmlimits@
}
\newcommand{\invlim}{%
  \mathop{\mathpalette\varlim@{\leftarrowfill@\scriptscriptstyle}}\nmlimits@
}
\makeatother

% Tor's stuff
\renewcommand{\i}{^{-1}}

\newcommand{\til}{\widetilde}


\newcommand{\sm}{\setminus}
\newcommand{\es}{\emptyset}
\newcommand{\se}{\subseteq}

\newcommand{\oc}[1]{\overset{\circ}{#1}}
\newcommand{\into}{\hookrightarrow}
\newcommand{\nb}{\nabla}
\renewcommand{\hat}{\widehat}

\usepackage{scalerel,stackengine}
\stackMath
\renewcommand\widehat[1]{%
\savestack{\tmpbox}{\stretchto{%
  \scaleto{%
    \scalerel*[\widthof{\ensuremath{#1}}]{\kern.1pt\mathchar"0362\kern.1pt}%
    {\rule{0ex}{\textheight}}%WIDTH-LIMITED CIRCUMFLEX
  }{\textheight}% 
}{2.4ex}}%
\stackon[-6.9pt]{#1}{\tmpbox}%
}


\newcommand{\diagSquare}[8]{
\begin{tikzpicture}[node distance=2cm, auto]
\node (a)              { $ #5 $ };
\node (b) [right of=a] { $ #6 $ };
\node (c) [below of=a] { $ #7 $ };
\node (d) [right of=c] { $ #8 $ };
\draw[-to] (a) to node { $ #1 $ } (b);
\draw[-to] (a) to node { $ #2 $ } (c);
\draw[-to] (b) to node { $ #3 $ } (d);
\draw[-to] (c) to node { $ #4 $ } (d);
\end{tikzpicture}}


\newcommand{\congto}{\xrightarrow{\cong}}
\newcommand{\incto}[1][]{ \xhookrightarrow{#1} }
\newcommand{\xto}[1]{\xrightarrow{#1}}


% Projective spaces
\def\RP{\mathbb{RP}}
\def\CP{\mathbb{CP}}
\def\HP{\mathbb{HP}}


% Commen Cathegories
\DeclareMathOperator{\Top}{Top}
\DeclareMathOperator{\Ab}{Ab}
\DeclareMathOperator{\Cat}{Cat}
\DeclareMathOperator{\CAT}{CAT}
\DeclareMathOperator{\Mod}{Mod}
\DeclareMathOperator{\Ring}{Ring}
\DeclareMathOperator{\Group}{Group}
\DeclareMathOperator{\Sets}{Sets}


% Homological Algebra
\DeclareMathOperator{\Hom}{Hom}
\DeclareMathOperator{\Tor}{Tor}
\DeclareMathOperator{\Ext}{Ext}


% Common Lie Groups 
\DeclareMathOperator{\GL}{GL}
\DeclareMathOperator{\SU}{SU}
\DeclareMathOperator{\U}{U}
\DeclareMathOperator{\Sp}{Sp}

% Maths Operators
\DeclareMathOperator{\id}{id}
\DeclareMathOperator{\pr}{pr}


\DeclareMathOperator{\Ker}{Ker}
\DeclareMathOperator{\im}{Im}


% Standard setts.
\def \N {\mathbb{N}}
\def \Z {\mathbb{Z}}
\def \Q {\mathbb{Q}}
\def \R {\mathbb{R}}
\def \C {\mathbb{C}}
\def \H {\mathbb{H}}

\def \E {\mathbb{E}}
\def \Z {\mathbb{Z}}
\def \I {\mathbb{I}}
\def \J {\mathbb{J}}

% Vector calculus.
\newcommand{\dif}[3][]{
	\ensuremath{\frac{d^{#1} {#2}}{d {#3}^{#1}}}}
\newcommand{\pdif}[3][]{
	\ensuremath{\frac{\partial^{#1} {#2}}{\partial {#3}^{#1}}}}

% Vectors and matricies.
\newcommand{\mat}[1]{\begin{matrix} #1 \end{matrix}}
\newcommand{\pmat}[1]{\begin{pmatrix} #1 \end{pmatrix}}
\newcommand{\bmat}[1]{\begin{bmatrix} #1 \end{bmatrix}}

% Add space around the argument.
\newcommand{\qq}[1]{\quad#1\quad}
\newcommand{\q}[1]{\:\:#1\:\:}

% Implications
\newcommand{\la}{\ensuremath{\Longleftarrow}}
\newcommand{\ra}{\ensuremath{\Longrightarrow}}
\newcommand{\lra}{\ensuremath{\Longleftrightarrow}}

\newcommand{\pwf}[1]{\begin{cases} #1 \end{cases}}
\newcommand{\tif}{\text{if}}
\newcommand{\tand}{\text{and}}

% Shorthand
\newcommand{\vphi}{\varphi}
\newcommand{\veps}{\varepsilon}

\newcommand{\<}[1]{\langle #1 \rangle}

% Notation
\newcommand{\wddef}[1]{\underline{#1}}
\newcommand{\pref}[1]{(\ref{#1})}


% Maths Operators
\theoremstyle{plain}
\theoremstyle{definition}
\newtheorem{thrm}{Theorem}[section]
\newtheorem{prop}[thrm]{Proposition}
\newtheorem{corol}[thrm]{Corollary}
\newtheorem{lemma}[thrm]{Lemma}

\newtheorem{defn}[thrm]{Definition}
\newtheorem{exmp}[thrm]{Example}
\newtheorem{clame}[thrm]{Clame}


\theoremstyle{remark}
% \newtheorem{remark}[thrm]{\normalfont\large\textit Remark}
\newtheorem{remark}[thrm]{Remark}
\newtheorem{note}[thrm]{Note}


\newSimpleHeaderEnvironment{exercise}{Exercise }

\def\A{\mathbb{A}}
\DeclareMathOperator{\rad}{rad}
\DeclareMathOperator{\tr}{tr}

\DeclareMathOperator{\Sh}{Sh}
\DeclareMathOperator{\op}{^{op}}
\DeclareMathOperator{\Id}{Id}
\DeclareMathOperator{\Ouv}{Ouv}

\DeclareMathOperator{\Ann}{Ann}
\DeclareMathOperator{\Supp}{Supp}
\DeclareMathOperator{\res}{res}
\newcommand{\B}{\mathcal{B}}
\newcommand{\F}{\mathcal{F}}
\newcommand{\G}{\mathcal{G}}

%%%%%%%% Content %%%%%%%%%%%%%%%%%%%%%%%%
\begin{document}
\mmaketitle

% \begin{exercise}[1]
% \end{exercise}

\begin{exercise}[2]
We have 
\begin{align*}
F &: \Sh (X) \to \Sh_\B(X) ; \quad \F \mapsto \F|_{\B\op} \\
G &: \Sh_\B (X) \to \Sh(X) ; \quad \F \mapsto (U \mapsto \invlim_{V \subset U, V
\in \B} \F(V)).
\end{align*}

We need to show that there exists natural isomorphisms $\eta : F \circ G \to
\Id_{\Sh_\B(X)}$ and $\xi : G \circ F \to \Id_{\Sh(X)}$.

For convenience we will write $\B_U = \{ V \in \B \q| V \subset U  \}$, then we have 
\begin{equation*}
F \circ G: \Sh_{\B}(X) \to \Sh_{\B}(X); \F \mapsto 
\left(U \mapsto \invlim_{V \in \B_U} \F(V) \right).  
\end{equation*}

Since $U \in \B$, in particular $U \in \B_U$, so $U$ defines a final object in
$\B_U$, therefore $\F(U)$ defines an initial in the diagram. So 
$\F(U)$ is the inverse limit up to unique isomorphism.

For the other composition, we have 
\begin{equation*}
 G \circ F: \Sh(X) \to \Sh(X); \F \mapsto 
\left(U \mapsto \invlim_{V \in \B_U} \F(V) \right), 
\end{equation*}
but now $U$ is not in general in $\B$. However, $\B_U$ defines a cover of $U$ by
open sets, so since $\F$ is a sheaf $\F(U)$ satisfy the universal property of
the following equalizer
\begin{equation}
\begin{tikzcd}
\F(U) \arrow[r] & \prod_{V \subset U, V \in \B} \F(V) 
\arrow[r, shift left, "\res^{V_1}_{V_1\cap V_2}"] 
\arrow[r, shift right, "\res^{V_1}_{V_1\cap V_2}"'] 
& \prod_{V_1, V_2 \subset U} \F(V) 
\end{tikzcd}
\end{equation}
We claim that this universal property implies that $\F(U)$ satisfy the universal
property of the inverse limit (in fact the two universal properties are the
equivalent, but we don't need this fact) and thus $\F(U)$ is the inverse limit
up to unique isomorphism.

To prove the claim consider $V, V' \in \B$ such that $V' \subset V \subset U$.
Then equaliser diagram implies that the following diagram commutes
\begin{equation*}
\begin{tikzcd}
\F(U) \arrow[r] \arrow[rd] & \F(V) \arrow[d, "\res^V_{V\cap V'}"] \\
& \F(V')
\end{tikzcd}
\end{equation*}
So $\F(U)$ is a cone. Suppose $K$ is another cone of the diagram. Then in
particular the diagram 
\begin{equation*}
\begin{tikzcd}
K \arrow[r] \arrow[d] & \F(V_1) \arrow[d, "\res^{V_1}_{V_1\cap V_2}"] \\
\F(V_2) \arrow[r, "\res^{V_1}_{V_1\cap V_2}"] & F(V_1 \cap V_2) 
\end{tikzcd}
\end{equation*}
commutes, where $V_1, V_2 \in \B_U$. So $K$ is an equalizer and thus factors
through $\F(U)$ by a unique map. 

These unique isomorphisms is the maps that will define the isomorphisms of sheaves
(ie. natural isomorphism) that will again define the natural isomorphisms
from the compositors $F\circ G$ and $G \circ F$ to the respective identities.
It remains to show the neutrality requirements these unique isomorphisms are
assemble into morphisms of sheaves that again assemble into natural
transformations. We will fist consider the composition $G \circ F$. 

%and then at the end observe that the other arguments in the case of $F \circ G$ follow.

Consider a pair $U' \subset U \subset X$ of open sets, and denote the
inclusion by $\iota : U' \hookrightarrow U$ and fix a sheaf $\F$. Then 
$\iota^* := (G \circ F)(\F)(\iota)$ is defined by the universal property of the
inverse limit.

Let $V',V \in \B_{U'}$, then the following diagram commutes:
\begin{equation}
\label{dg:1}
\begin{tikzcd}
\invlim\limits_{W \in \B_U} \F(W) \arrow[r] \arrow[rd] 
\arrow[rr, bend left=20, dotted, "\iota^*"]
& \F(V) \arrow[d] 
& \invlim\limits_{W \in \B_U'} \F(W) \arrow[l] \arrow[ld] \\
& \F(V')
\end{tikzcd} 
\end{equation}
% Here the right triangle is the defining diagram for of the limit over $\B_{U'}$
% and the left triangle it the defining diagram for the limit over $\B_U$. 
Since $\B_{U'} \subset \B_U$, the left triangle holds for all $V,V' \in \B_{U'}$, so 
the limit over $\B_U$ is a cone over the diagram indexed by $\B_{U'}$.
So by the universal property of the limit over $\B_{U'}$, 
there is a unique map a $\iota^*$, such that the diagram commutes, for all $V,
V' \in \B_{U'}$.

Now that we have this map, we want to show that the diagram
\begin{equation}
\label{dg:2}
\begin{tikzcd}
\invlim\limits_{W \in \B_U} \F(W) \arrow[r, "\iota^*"] \arrow[d, "\cong"]
& \invlim\limits_{W \in \B_U'} \F(W) \arrow[d, "\cong"] \\
\F(U) \arrow[r, "\F(\iota)"] & \F(U')
\end{tikzcd}
\end{equation}
commutes, which shows that the unique isomorphism define a morphism of sheaves. 
We will denote the isomorphism on the left by $\psi$, the one on the right by
$\phi$, the restriction maps from $\invlim_{W \in \B_U} \F(W) \to \F(W)$ by
$\rho_W$ and from $\invlim_{W \in \B_U'} \F(W) \to \F(W)$ by $\rho'_W$. Then by
the defining properties of $\psi$ and $\phi$, 
\begin{equation}
\label{eq:1}
\rho_W \circ \psi^{-1} = \res^U_W \qq\tand \rho'_W \circ \phi^{-1} =
\res^{U'}_W. 
\end{equation}
So since $\F$ is a sheaf $\rho'_W \circ \phi^{-1} \circ \F(\iota) = \rho_W \circ
\psi^{-1}$ (it follows from the equalizer condition for the cover
$\{ U, U' \}$ of $U$.) This implies that diagram \pref{dg:2} commutes, by
considering the diagram combined with diagram \pref{dg:1}.



\newcommand{\ha}{\hat{\alpha}}

Let $\F, \G : \Ouv(X)\op \to \Sets$ be sheaves and $\alpha: \F \to \G$ a
morphism of sheaves. Then like above $\ha = (G \circ F)(\alpha)$ is defined by
the universal property of the inverse limit.
Let $U\subset X$ open, then for all $V' \subset V$, the diagram
\begin{equation}
\begin{tikzcd}
\label{eq:5}
\invlim\limits_{V \in \B_U} \F(V) \arrow[r] \arrow[rrr, bend left=20, dotted, "\ha_U"]
\arrow[rd]
& \F(V) \arrow[r, "\alpha_V"] \arrow[d]
& \G(V) \arrow[d]
& \invlim\limits_{V \in \B_U} \G(V) \arrow[l] \arrow[ld] \\
& \F(V') \arrow[r, "\alpha_{V'}"]
& \G(V')
\end{tikzcd} 
\end{equation}
without $\ha_U$ commutes. So $\invlim_{V \subset U} \F(V)$ is a cone over $\G(V)$ indexed by
$\B_U$ and $\ha_U$ is the unique map such that the diagram commutes. 
To show that $\ha_U$ assemble into a morphism of sheaves we need the outer
square in the following diagram to commute

\renewcommand{\L}[2]{\invlim\limits_{V \in \B_{#2}} #1(V)}

\begin{equation*}
\begin{tikzcd}
 \L{\F}{U} \arrow[rrr, "\ha_U"] \arrow[dd, "\iota^*"] \arrow[rd] &&& 
 \L{\G}{U} \arrow[dd, "\iota^*"] \arrow[ld]  \\
 & \F(V) \arrow[r, "\alpha_V"] & \G(V) \\
 \L{\F}{U'} \arrow[rrr, "\ha_{U'}"] \arrow[ru] &&& 
\L{\G}{U'} \arrow[lu]
\end{tikzcd} 
\end{equation*}

We obeserve that the left and right triangles comutes by the definition of
$\iota^*$ and that the top and bottum inner squares cummutes by the defention of
$\ha$. So in particular, all paths from $\invlim_{V \in \B_U} \F(V)$ to
$\G(V)$ agree and thus the outer square most commute by the universal propery of
the limit at the bottum right.

%U \subset X$ be open. 

The only thing that remains to show now is that the following diagram commutes

\begin{equation*}
\begin{tikzcd}
\L{\F}{U} \arrow[r,"\ha_U"] \arrow[d, "\psi"] & \L{G}{U} \arrow[d, "\phi"]\\
\F(U) \arrow[r, "\alpha_U"] & \G(U)
\end{tikzcd}
\end{equation*}

Similarly to how we showed diagram \pref{dg:2}, by combining the diagram with
diagram \pref{dg:5} and use equation \pref{eq:1}, we get that 
$\tau_V \circ \phi^{-1} \circ \alpha_U$ where $\tau_V$ is the restiction of the
limit onto $\G(V)$ is the same as $\alpha_V \circ \rho_V \circ \psi^{-1}$. Since
$\alpha_V \circ \rho_V$, for $V\in \B_U$, are the maps of the cone in diagram
$\pref{dg:5}$,
\[ \alpha_V \circ \rho_V = \tau_V \circ \ha_U. \]
So 
\[ \tau_V \circ \phi^{-1} \circ \alpha_U = \tau_V\circ \ha_U  \circ \psi^{-1}. \]
Finaly, since this holds for all $V \in \B_U$, the univeral propery of the
inverse limit implies that we can cancele the $\tau_V$ ans thus the diaram
comutes.
\end{exercise}

\newcommand{\tM}{\tilde{M}}
\newcommand{\tm}{\tilde{m}}

\newpage
\begin{exercise}[3]
\begin{enumerate}
\item % 1
To show that $\Supp(s)$ closed, we will prove that $U \setminus \Supp(s)$ is
open. Suppose $x \notin \Supp(s)$, that is $s_x = 0$ or eqvivalently there 
exists $x \in V \subset U$ open such that $s|_V = 0$. But then $s_y = 0$ for all
$y\in V$. Hence $V \cap \Supp(s) = \emptyset$, which proves that $U \setminus
\Supp(s)$ is open and thus $\Supp(s)$ is closed in $U$.


\item %2
For $f\in A$, $\tM := M \otimes_A A[f^{-1}] \cong M[f^{-1}]$. 

We have 
\begin{align*}
\Supp(\tm) &:= \{ x \in X \q| \tm_x \ne 0 \}  \\
V(\Ann_A(m)) &:= \{ x \in X \q| \forall f \in \Ann_A(m) \q, f(x) = 0  \} 
\end{align*}

Like in part 1. we will consider the complement of the sets and show that they
are the same. We have
\begin{align*}
% X\setminus \Supp(\tm) &:= \{ x \in X \q| \tm_x \ne 0 \}  \\
% X\setminus V(\Ann_A(m)) &:= \{ x \in X \q| \forall f \in \Ann_A(m) \q, f(x) = 0  \} 
x \notin \Supp(\tm) &\q\lra \exists f \in A \q: x \in D(f) \q\land (\tm, D(f)) = 0 \\
x \notin V(\Ann_A(m)) &\q\lra \exists f \in \Ann_A(m)\q: f(x) \ne 0 \\
\end{align*}

Suppose $x \notin \Supp(\tm)$ and let $f \in \Ann_A(m)$ such that $f(x) \ne 0$.
Then $f m = 0$ and $x \in D(f)$, so $(\tm, D(f)) = 0$ and thus $x \notin
\Supp(\tm)$.

Conversly suppose $x \notin \Supp(\tm)$ and let $f \in A$ such that $x \in
D(f)$ and $(\tm, D(f)) = 0$. Then $f^n m = 0$ for some $n \in \N$, so $f^n \in
\Ann_A(m)$. Since $x \in D(f)$, $f(x) \ne 0$ so $f \not\subseteq x$. Therefore 
$f^n \notin x$, since $x$ is a prime ideal, and thus $f^n(x) \ne 0$. So $g = f^n
\in \Ann_A(m)$ and $x \in D(g)$. Hence $x \notin V(\Ann_A(m))$.

Now suppose $M = \left<m_1, ..., m_n\right>$. 
Suppose $x \notin V(\Ann_A(M))$, that is there exists $f \in \Ann_A(M)$ such
that $f(x) \ne 0$. So $fm = 0$ for all $m \in M$ and thus,
by the argument above, $(\tm, D(f)) = 0$ and $x\in D(f)$. Hence $\tM_x = 0$ and
thus $x \notin \Supp(\tM)$.

Conversly suppose $x \notin \Supp(\tM)$, that is $\tM_x = 0$. So there exits $f
\in A$ such that $x \in D(f)$ and for all $\tm \in M[f^{-1}]$ $(\tm, D(f)) =
0$. In particular $(\tm_i, D(f)) = 0$, so by the argument above there exists
$f_i \in A$ such that $f_i m_i = 0$ and $f_i(x) \ne 0$. 
Let $g = \prod_i f_i$ and consider $m \in M$, then $m = \sum_i a_i m_i$, so
$gm = \sum_i a_i \prod_j f_j m_i = 0$ and ths $g \in \Ann_A(M)$. 
Furthermore $f_i \notin x$, so since $x$ is prime, $g \notin x$. Hence 
$x \notin \Supp(\tM)$.

\item %3
Let $A = \Z$ and $M = \bigoplus_{p \text{ prime}} \Z_p$. Consider $p \in \N$ be prime
and $n \in \Z$ such that $p \in D(n) = \{ q\in X \q| q \nmid n \}$ (ie. $p \nmid
n$). Let $m = (m_i)_{i \text{ prime}}$ and $m_i = 1$, then $nm \ne 0$, since
$n m_p = 0$ would imply $p \mid n$. So $p \in \Supp(\tM)$. 

On the other hand $0 \notin \Supp(\tM)$. Since if $0 \in D(n)$, then $n \in
\left<0\right>$, so $n = 0$ and thus $nm = 0$ for all $m\in M$.

So $\Supp(\tM) = X \setminus \{0\}$. But $\{0\}$ is not closed in $X$, since the
only closed set contiaining $0$ is $X$.
\end{enumerate}
\end{exercise}



\end{document}
