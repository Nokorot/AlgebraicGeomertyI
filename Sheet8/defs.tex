

\newsavebox{\pullback}
\sbox\pullback{%
\begin{tikzpicture}%
\draw (0,0) -- (1ex,0ex) -- (1ex, 1ex);
\end{tikzpicture}}


% Algebraic Geometry stuff
\def\A{\mathbb{A}}
\DeclareMathOperator{\rad}{rad}
\DeclareMathOperator{\tr}{tr}

\DeclareMathOperator{\Sh}{Sh}
\DeclareMathOperator{\op}{^{op}}
\DeclareMathOperator{\Id}{Id}
\DeclareMathOperator{\Ouv}{Ouv}

\DeclareMathOperator{\Ann}{Ann}
\DeclareMathOperator{\Supp}{Supp}
\DeclareMathOperator{\res}{res}
\newcommand{\B}{\mathcal{B}}
\newcommand{\F}{\mathcal{F}}
\newcommand{\G}{\mathcal{G}}


% Direct limit and colimit
\setlength{\parindent}{0cm}
\setlength{\parskip}{3pt}

\makeatletter
\newcommand{\lim@}[2]{%
  \vtop{\m@th\ialign{##\cr
    \hfil$#1\operator@font lim$\hfil\cr
    \noalign{\nointerlineskip\kern1.5\ex@}#2\cr
    \noalign{\nointerlineskip\kern-\ex@}\cr}}%
}
\newcommand{\colim}{%
  \mathop{\mathpalette\varlim@{\rightarrowfill@\scriptscriptstyle}}\nmlimits@
}
\newcommand{\invlim}{%
  \mathop{\mathpalette\varlim@{\leftarrowfill@\scriptscriptstyle}}\nmlimits@
}
\makeatother

% Tor's stuff
\renewcommand{\i}{^{-1}}

\newcommand{\til}{\widetilde}


\newcommand{\sm}{\setminus}
\newcommand{\es}{\emptyset}
\newcommand{\se}{\subseteq}

\newcommand{\oc}[1]{\overset{\circ}{#1}}
\newcommand{\into}{\hookrightarrow}
\newcommand{\nb}{\nabla}
\renewcommand{\hat}{\widehat}

\usepackage{scalerel,stackengine}
\stackMath
\renewcommand\widehat[1]{%
\savestack{\tmpbox}{\stretchto{%
  \scaleto{%
    \scalerel*[\widthof{\ensuremath{#1}}]{\kern.1pt\mathchar"0362\kern.1pt}%
    {\rule{0ex}{\textheight}}%WIDTH-LIMITED CIRCUMFLEX
  }{\textheight}% 
}{2.4ex}}%
\stackon[-6.9pt]{#1}{\tmpbox}%
}


\newcommand{\diagSquare}[8]{
\begin{tikzpicture}[node distance=2cm, auto]
\node (a)              { $ #5 $ };
\node (b) [right of=a] { $ #6 $ };
\node (c) [below of=a] { $ #7 $ };
\node (d) [right of=c] { $ #8 $ };
\draw[-to] (a) to node { $ #1 $ } (b);
\draw[-to] (a) to node { $ #2 $ } (c);
\draw[-to] (b) to node { $ #3 $ } (d);
\draw[-to] (c) to node { $ #4 $ } (d);
\end{tikzpicture}}


\newcommand{\congto}{\xrightarrow{\cong}}
\newcommand{\incto}[1][]{ \xhookrightarrow{#1} }
\newcommand{\xto}[1]{\xrightarrow{#1}}


% Projective spaces
\def\RP{\mathbb{RP}}
\def\CP{\mathbb{CP}}
\def\HP{\mathbb{HP}}


% Commen Cathegories
\DeclareMathOperator{\Top}{Top}
\DeclareMathOperator{\Ab}{Ab}
\DeclareMathOperator{\Cat}{Cat}
\DeclareMathOperator{\CAT}{CAT}
\DeclareMathOperator{\Mod}{Mod}
\DeclareMathOperator{\Ring}{Ring}
\DeclareMathOperator{\Group}{Group}
\DeclareMathOperator{\Sets}{Sets}


% Homological Algebra
\DeclareMathOperator{\Hom}{Hom}
\DeclareMathOperator{\Tor}{Tor}
\DeclareMathOperator{\Ext}{Ext}


% Common Lie Groups 
\DeclareMathOperator{\GL}{GL}
\DeclareMathOperator{\SU}{SU}
\DeclareMathOperator{\U}{U}
\DeclareMathOperator{\Sp}{Sp}

% Maths Operators
\DeclareMathOperator{\id}{id}
\DeclareMathOperator{\pr}{pr}


\DeclareMathOperator{\Ker}{Ker}
\DeclareMathOperator{\im}{Im}


% Standard setts.
\def \N {\mathbb{N}}
\def \Z {\mathbb{Z}}
\def \Q {\mathbb{Q}}
\def \R {\mathbb{R}}
\def \C {\mathbb{C}}
\def \H {\mathbb{H}}

\def \E {\mathbb{E}}
\def \Z {\mathbb{Z}}
\def \I {\mathbb{I}}
\def \J {\mathbb{J}}

% Vector calculus.
\newcommand{\dif}[3][]{
	\ensuremath{\frac{d^{#1} {#2}}{d {#3}^{#1}}}}
\newcommand{\pdif}[3][]{
	\ensuremath{\frac{\partial^{#1} {#2}}{\partial {#3}^{#1}}}}

% Vectors and matricies.
\newcommand{\mat}[1]{\begin{matrix} #1 \end{matrix}}
\newcommand{\pmat}[1]{\begin{pmatrix} #1 \end{pmatrix}}
\newcommand{\bmat}[1]{\begin{bmatrix} #1 \end{bmatrix}}

% Add space around the argument.
\newcommand{\qq}[1]{\quad#1\quad}
\newcommand{\q}[1]{\:\:#1\:\:}

% Implications
\newcommand{\la}{\ensuremath{\Longleftarrow}}
\newcommand{\ra}{\ensuremath{\Longrightarrow}}
\newcommand{\lra}{\ensuremath{\Longleftrightarrow}}

\newcommand{\pwf}[1]{\begin{cases} #1 \end{cases}}
\newcommand{\tif}{\text{if}}
\newcommand{\tand}{\text{and}}

% Shorthand
\newcommand{\vphi}{\varphi}
\newcommand{\veps}{\varepsilon}

\newcommand{\<}[1]{\langle #1 \rangle}

% Notation
\newcommand{\wddef}[1]{\underline{#1}}
\newcommand{\pref}[1]{(\ref{#1})}


% Maths Operators
\theoremstyle{plain}
\theoremstyle{definition}
\newtheorem{thrm}{Theorem}[section]
\newtheorem{prop}[thrm]{Proposition}
\newtheorem{corol}[thrm]{Corollary}
\newtheorem{lemma}[thrm]{Lemma}

\newtheorem{defn}[thrm]{Definition}
\newtheorem{exmp}[thrm]{Example}
\newtheorem{clame}[thrm]{Clame}


\theoremstyle{remark}
% \newtheorem{remark}[thrm]{\normalfont\large\textit Remark}
\newtheorem{remark}[thrm]{Remark}
\newtheorem{note}[thrm]{Note}
