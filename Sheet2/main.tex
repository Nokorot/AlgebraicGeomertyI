\documentclass[a4paper,11pt,english]{article}
\usepackage{.styles/basic}
\usepackage{.styles/envs}

%%%%%%% Title %%%%%%%%%%%%%%%%%%%%%%%%%%%
\title{\textbf{Algebraic Geometry} - Exercise Sheet 2}
\author{Tor Gjone \& Paul}

%%%%%%% Definitions %%%%%%%%%%%%%%%%%%%%%


\newsavebox{\pullback}
\sbox\pullback{%
\begin{tikzpicture}%
\draw (0,0) -- (1ex,0ex) -- (1ex, 1ex);
\end{tikzpicture}}


% Algebraic Geometry stuff
\def\A{\mathbb{A}}
\DeclareMathOperator{\rad}{rad}
\DeclareMathOperator{\tr}{tr}

\DeclareMathOperator{\Sh}{Sh}
\DeclareMathOperator{\op}{^{op}}
\DeclareMathOperator{\Id}{Id}
\DeclareMathOperator{\Ouv}{Ouv}

\DeclareMathOperator{\Ann}{Ann}
\DeclareMathOperator{\Supp}{Supp}
\DeclareMathOperator{\res}{res}
\newcommand{\B}{\mathcal{B}}
\newcommand{\F}{\mathcal{F}}
\newcommand{\G}{\mathcal{G}}


% Direct limit and colimit
\setlength{\parindent}{0cm}
\setlength{\parskip}{3pt}

\makeatletter
\newcommand{\lim@}[2]{%
  \vtop{\m@th\ialign{##\cr
    \hfil$#1\operator@font lim$\hfil\cr
    \noalign{\nointerlineskip\kern1.5\ex@}#2\cr
    \noalign{\nointerlineskip\kern-\ex@}\cr}}%
}
\newcommand{\colim}{%
  \mathop{\mathpalette\varlim@{\rightarrowfill@\scriptscriptstyle}}\nmlimits@
}
\newcommand{\invlim}{%
  \mathop{\mathpalette\varlim@{\leftarrowfill@\scriptscriptstyle}}\nmlimits@
}
\makeatother

% Tor's stuff
\renewcommand{\i}{^{-1}}

\newcommand{\til}{\widetilde}


\newcommand{\sm}{\setminus}
\newcommand{\es}{\emptyset}
\newcommand{\se}{\subseteq}

\newcommand{\oc}[1]{\overset{\circ}{#1}}
\newcommand{\into}{\hookrightarrow}
\newcommand{\nb}{\nabla}
\renewcommand{\hat}{\widehat}

\usepackage{scalerel,stackengine}
\stackMath
\renewcommand\widehat[1]{%
\savestack{\tmpbox}{\stretchto{%
  \scaleto{%
    \scalerel*[\widthof{\ensuremath{#1}}]{\kern.1pt\mathchar"0362\kern.1pt}%
    {\rule{0ex}{\textheight}}%WIDTH-LIMITED CIRCUMFLEX
  }{\textheight}% 
}{2.4ex}}%
\stackon[-6.9pt]{#1}{\tmpbox}%
}


\newcommand{\diagSquare}[8]{
\begin{tikzpicture}[node distance=2cm, auto]
\node (a)              { $ #5 $ };
\node (b) [right of=a] { $ #6 $ };
\node (c) [below of=a] { $ #7 $ };
\node (d) [right of=c] { $ #8 $ };
\draw[-to] (a) to node { $ #1 $ } (b);
\draw[-to] (a) to node { $ #2 $ } (c);
\draw[-to] (b) to node { $ #3 $ } (d);
\draw[-to] (c) to node { $ #4 $ } (d);
\end{tikzpicture}}


\newcommand{\congto}{\xrightarrow{\cong}}
\newcommand{\incto}[1][]{ \xhookrightarrow{#1} }
\newcommand{\xto}[1]{\xrightarrow{#1}}


% Projective spaces
\def\RP{\mathbb{RP}}
\def\CP{\mathbb{CP}}
\def\HP{\mathbb{HP}}


% Commen Cathegories
\DeclareMathOperator{\Top}{Top}
\DeclareMathOperator{\Ab}{Ab}
\DeclareMathOperator{\Cat}{Cat}
\DeclareMathOperator{\CAT}{CAT}
\DeclareMathOperator{\Mod}{Mod}
\DeclareMathOperator{\Ring}{Ring}
\DeclareMathOperator{\Group}{Group}
\DeclareMathOperator{\Sets}{Sets}


% Homological Algebra
\DeclareMathOperator{\Hom}{Hom}
\DeclareMathOperator{\Tor}{Tor}
\DeclareMathOperator{\Ext}{Ext}


% Common Lie Groups 
\DeclareMathOperator{\GL}{GL}
\DeclareMathOperator{\SU}{SU}
\DeclareMathOperator{\U}{U}
\DeclareMathOperator{\Sp}{Sp}

% Maths Operators
\DeclareMathOperator{\id}{id}
\DeclareMathOperator{\pr}{pr}


\DeclareMathOperator{\Ker}{Ker}
\DeclareMathOperator{\im}{Im}


% Standard setts.
\def \N {\mathbb{N}}
\def \Z {\mathbb{Z}}
\def \Q {\mathbb{Q}}
\def \R {\mathbb{R}}
\def \C {\mathbb{C}}
\def \H {\mathbb{H}}

\def \E {\mathbb{E}}
\def \Z {\mathbb{Z}}
\def \I {\mathbb{I}}
\def \J {\mathbb{J}}

% Vector calculus.
\newcommand{\dif}[3][]{
	\ensuremath{\frac{d^{#1} {#2}}{d {#3}^{#1}}}}
\newcommand{\pdif}[3][]{
	\ensuremath{\frac{\partial^{#1} {#2}}{\partial {#3}^{#1}}}}

% Vectors and matricies.
\newcommand{\mat}[1]{\begin{matrix} #1 \end{matrix}}
\newcommand{\pmat}[1]{\begin{pmatrix} #1 \end{pmatrix}}
\newcommand{\bmat}[1]{\begin{bmatrix} #1 \end{bmatrix}}

% Add space around the argument.
\newcommand{\qq}[1]{\quad#1\quad}
\newcommand{\q}[1]{\:\:#1\:\:}

% Implications
\newcommand{\la}{\ensuremath{\Longleftarrow}}
\newcommand{\ra}{\ensuremath{\Longrightarrow}}
\newcommand{\lra}{\ensuremath{\Longleftrightarrow}}

\newcommand{\pwf}[1]{\begin{cases} #1 \end{cases}}
\newcommand{\tif}{\text{if}}
\newcommand{\tand}{\text{and}}

% Shorthand
\newcommand{\vphi}{\varphi}
\newcommand{\veps}{\varepsilon}

\newcommand{\<}[1]{\langle #1 \rangle}

% Notation
\newcommand{\wddef}[1]{\underline{#1}}
\newcommand{\pref}[1]{(\ref{#1})}


% Maths Operators
\theoremstyle{plain}
\theoremstyle{definition}
\newtheorem{thrm}{Theorem}[section]
\newtheorem{prop}[thrm]{Proposition}
\newtheorem{corol}[thrm]{Corollary}
\newtheorem{lemma}[thrm]{Lemma}

\newtheorem{defn}[thrm]{Definition}
\newtheorem{exmp}[thrm]{Example}
\newtheorem{clame}[thrm]{Clame}


\theoremstyle{remark}
% \newtheorem{remark}[thrm]{\normalfont\large\textit Remark}
\newtheorem{remark}[thrm]{Remark}
\newtheorem{note}[thrm]{Note}


\newSimpleHeaderEnvironment{exercise}{Exercise }

\def\A{\mathbb{A}}
\DeclareMathOperator{\rad}{rad}
\DeclareMathOperator{\tr}{tr}

\setlength{\parindent}{0cm}
\setlength{\parskip}{3pt}

\makeatletter
\newcommand{\colim@}[2]{%
  \vtop{\m@th\ialign{##\cr
    \hfil$#1\operator@font lim$\hfil\cr
    \noalign{\nointerlineskip\kern1.5\ex@}#2\cr
    \noalign{\nointerlineskip\kern-\ex@}\cr}}%
}
\newcommand{\colim}{%
  \mathop{\mathpalette\colim@{\rightarrowfill@\textstyle}}\nmlimits@
}
\newcommand{\invlim@}[2]{%
  \vtop{\m@th\ialign{##\cr
    \hfil$#1\operator@font lim$\hfil\cr
    \noalign{\nointerlineskip\kern1.5\ex@}#2\cr
    \noalign{\nointerlineskip\kern-\ex@}\cr}}%
}
\newcommand{\invlim}{%
  \mathop{\mathpalette\colim@{\leftarrowfill@\textstyle}}\nmlimits@
}


\makeatother

%%%%%%%% Content %%%%%%%%%%%%%%%%%%%%%%%%
\begin{document}
\mmaketitle

\begin{exercise}[2]
\begin{enumerate}
\item
% This is quite clear. 
Suppose $(U_i)_{i\in \N}$ is some ascending chain of open sets.
Then $(U_i)_{i\in \N}$ defines a cover of the open set $U = \cup_{i\in \N} U_i$,
which is also open and thus quasi-compact. So there is some $i_0 \in \N$ such
that $U_{i_0} = U$ and thus $U_i = U_{i_0}$ for all $i\ge i_0$.

Conversely, if $X$ is noetherian and $(U_j)_{j\in J}$ is a cover of $U$. 

Let $I \subset \P(J)$, be the set consisting of finite number of elements in
$J$ and define $V_i = \cup_{j \in i} U_j$. Then $I$ we define a partial order on
$I$ by $i_1 \le i_2$ if and only if $V_{i_1} \subset V_{i_2}$ 
( Note: that $I$ also inherits a partial order from $\P(J)$, and convivially 
$i_1 \subset i_2$ does imply $i_1 \le i_2$, but the converse is not true in
general.)
Observe that since $X$ is assumed to be noetherian, $I$ satisfy the ascending
chain condition, since an ascending chain in $I$ defines, in particular an
ascending chain of open sets $X$ (and the partial orders agree, by construction.) 

So, by Zorn's lemma, there exist a maximal element $i_0 \in I$, that is a finite
subset of $J$. Since $i_0$ is maximal, in fact, $U_j \subset V_{i_0}$ for all
$j\in J$. Suppose, for contradiction, that some $U_j$ is not contained in
$V_{i_0}$, then $V_{i_0} \subsetneq V_{i_0 \cup {j}}$, a contradiction. So
$U = \cup_{j \in J} \subset V_{i_0} = \cup_{j \in i_0} U_j$.

\item

\item
In a noetherian ring, every ideal is finitely generated and every open subset 
in $\Spec(A)$ is of the form $\Spec(A)\setminus V(I)$ for some ideal. So every 
open subset of $\Spec(A)$ is quasi-compact.

\item
Consider $A = \Z$, with a trivial product, that is $a b = 0$ for all $a,b \in A$.
Then every subset of $A$ defines an ideal and every ideal is a subset of the
nilradical. So $\Spec(A) = \Spec(\{0\}) = \{0\}$. There is only one possible
topology on one point and this topology is quasi-compact. Bu $A$ is clearly not
noetherian, since the ascending chain of ideals defined by
$(\{0,1,...,n\})_{n\in \N}$ is infinite and strictly increasing.
\end{enumerate}
\end{exercise} 

\begin{exercise}[4]
Let $\iota^A_{i,j}$ denote the map $A_i \to A_j$. We have 
\[ \colim_{I} A_i = \sqcup_{i\in I} / \sim, \]
where $(a,i) \sim (b,j)$ iff $\iota^A_{i,k} (a) = \iota^A_{j,k}(b)$ for some 
$k \ge i,j$.

We will by $\alpha$ denote the map from $\colim_I A_i$ to $\colim_I B_i$, defined
by the universal property of the colimit and the maps $\alpha_i$. That is 
$\alpha$ is the unique map map defined by the colimit such that 
$\alpha \circ \itoa^A_i = \itoa^B_i \circ \alpha_i$, where $\itoa^A_i$ is the
unique map defined by the colimit such that $\iota^A_i = \itoa^A_j \circ
\iota^A_{i,j}$ for all $j \ge i$. 
Similarly $\beta$ denotes the map from $\colim_I B_i$ to $\colim_I C_i$.

There are four parts to showing that the sequence of colimits is exact. 
\begin{enumerate}
\item $\alpha$ is injective:

Suppose $\alpha(a, i) = \alpha(b,j) = (c,l)_B$ for some $a \in A_i$, $b \in A_j$
and $c \in B_l$. Let $k \ge i,j,l$. We have

\begin{align*}
\alpha(a,i) &= (\alpha \circ \itoa^A_i)(a) \\
&= (\alpha \circ \iota^A_k \circ \itoa^A_{i,k}) (a) \\
&= (\iota^B_k \circ \alpha_k \circ \itoa^A_{i,k}) (a),
\quad \text{and similarly} \\
\alpha(b,j) 
&= (\iota^B_k \circ \alpha_k \circ \itoa^A_{j,k}) (b).
\end{align*}
So $\alpha_k \circ \itoa^A_{i,k} (a) \sim \alpha_k \circ \itoa^A_{j,k} (b)$.
That is
\[ \iota^B_\{ k, m \} \circ \alpha_k \circ \itoa^A_{i,k} (a) 
= \iota^B_\{ k, m \} \circ \alpha_k \circ \itoa^A_{j,k} (b), \]
for some $m \ge k$. Rearranging one more time
\[ \iota^B_\{ k, m \} \circ \alpha_k \circ \itoa^A_{i,k} 
= \alpha_m \circ \iota^B_\{ k, m \} \circ  \itoa^A_{i,k}
= \alpha_m \circ \iota^b_\{ i, m \} \]
So since $\alpha_m$ is injective, we have $\iota^b_\{ i, m \}(a) = \iota^b_\{ j,
m \}(b)$. Hence $(a,i) \sim (b,j)$.

\[ \alpha_m \circ \iota^B_\{ k, m \} \circ \itoa^A_{i,k} (a) 
= \iota^B_\{ k, m \} \circ \alpha_k \circ \itoa^A_{j,k} (b), \]

\item $\beta$ is surjective:

Let $(c,i) \in \colim_I C_i$, then, since $\beta_i$ is surjective,
there exists $b\in B_i$ such that $\beta_i(b) = c$. So 
\[ \beta(b,i) = (\bata \circ \iota^B_i)(b) = (\iota^C_i \circ \beta_i)(b) 
=  \iota^C_i (c) = (c,i). \]

\item $\alpha \circ \beta = 0$:

Let $(a,i) \in \colim_I A_i$, then 
\begin{align*}
(\beta \circ \alpha)(a,i) &= (\beta \circ \alpha \circ \iota^A_i) (a) \\
&= (\beta \circ \iota^B_i \circ \alpha_i)(a) \\
&= (\iota^C_i \circ \beta_i \circ \alpha_i)(a) = 0.
\end{align*}

\item $\ker \beta \subset \im \alpha$:

Suppose $(b,i) \in \ker(\beta)$, then
\begin{align*}
0 = \beta(b,i) = (\beta \circ \iota^B_i)(b) = (\iota^C_i \circ \beta_i),
\end{align*}
that is $(\beta_i(b),i) \sim (0,j)$ for some $j\in I$. Let $k \ge i,j$ such that 
$\itoa^C_{i,k}(\beta_i(b)) = \iota^C_{j,k}(0) = 0$ (where the last equality
comes from $\iota^C_{j,k}$ being an homomorphism).
Since $\itoa^C_{i,k}(\beta_i(b)) = \beta_k \circ \iota^B_{i,k} (b)$, we have
$\iota^B_{i,k} (b) \in \ker \beta_k$. So there exists some $a \in A_k$ such that
$\alpha_k(a) = \iota^B_{i,k} (b)$, since $\ker(\beta_k) = \im(\alpha_k)$, by
assumption. Hence 
\[ \alpha(a,k) = (\alpha \circ \iota^A_k)(a) = (\iota^B_k \circ \alpha_k)(a) 
= (\iota^B_k \circ \iota^B_{i,k})(b) = \iota^B_i(b) = (b,i). \]



\end{enumerate}









\end{exercise}


\end{document}
