\documentclass[a4paper,11pt,english]{article}
\usepackage{.styles/basic}
\usepackage{.styles/envs}

%%%%%%% Title %%%%%%%%%%%%%%%%%%%%%%%%%%%
\title{\textbf{Algebraic Geometry} - Exercise Sheet 1}
\author{Tor Gjone \& Paul}

%%%%%%% Definitions %%%%%%%%%%%%%%%%%%%%%


% Algebraic Geometry stuff
\def\A{\mathbb{A}}
\DeclareMathOperator{\rad}{rad}
\DeclareMathOperator{\tr}{tr}

\DeclareMathOperator{\Sh}{Sh}
\DeclareMathOperator{\op}{^{op}}
\DeclareMathOperator{\Id}{Id}
\DeclareMathOperator{\Ouv}{Ouv}

\DeclareMathOperator{\Ann}{Ann}
\DeclareMathOperator{\Supp}{Supp}
\DeclareMathOperator{\res}{res}
\newcommand{\B}{\mathcal{B}}
\newcommand{\F}{\mathcal{F}}
\newcommand{\G}{\mathcal{G}}


% Direct limit and colimit
\setlength{\parindent}{0cm}
\setlength{\parskip}{3pt}

\makeatletter
\newcommand{\lim@}[2]{%
  \vtop{\m@th\ialign{##\cr
    \hfil$#1\operator@font lim$\hfil\cr
    \noalign{\nointerlineskip\kern1.5\ex@}#2\cr
    \noalign{\nointerlineskip\kern-\ex@}\cr}}%
}
\newcommand{\colim}{%
  \mathop{\mathpalette\varlim@{\rightarrowfill@\scriptscriptstyle}}\nmlimits@
}
\newcommand{\invlim}{%
  \mathop{\mathpalette\varlim@{\leftarrowfill@\scriptscriptstyle}}\nmlimits@
}
\makeatother

% Tor's stuff
\renewcommand{\i}{^{-1}}

\newcommand{\til}{\tilde}


\newcommand{\sm}{\setminus}
\newcommand{\es}{\emptyset}
\newcommand{\se}{\subseteq}

\newcommand{\oc}[1]{\overset{\circ}{#1}}
\newcommand{\into}{\hookrightarrow}
\newcommand{\nb}{\nabla}
\renewcommand{\hat}{\widehat}

\usepackage{scalerel,stackengine}
\stackMath
\renewcommand\widehat[1]{%
\savestack{\tmpbox}{\stretchto{%
  \scaleto{%
    \scalerel*[\widthof{\ensuremath{#1}}]{\kern.1pt\mathchar"0362\kern.1pt}%
    {\rule{0ex}{\textheight}}%WIDTH-LIMITED CIRCUMFLEX
  }{\textheight}% 
}{2.4ex}}%
\stackon[-6.9pt]{#1}{\tmpbox}%
}




\newcommand{\diagSquare}[8]{
\begin{tikzpicture}[node distance=2cm, auto]
\node (a)              { $ #5 $ };
\node (b) [right of=a] { $ #6 $ };
\node (c) [below of=a] { $ #7 $ };
\node (d) [right of=c] { $ #8 $ };
\draw[-to] (a) to node { $ #1 $ } (b);
\draw[-to] (a) to node { $ #2 $ } (c);
\draw[-to] (b) to node { $ #3 $ } (d);
\draw[-to] (c) to node { $ #4 $ } (d);
\end{tikzpicture}
}


\newcommand{\congto}{\xrightarrow{\cong}}
\newcommand{\incto}[1][]{ \xhookrightarrow{#1} }
\newcommand{\xto}[1]{\xrightarrow{#1}}


% Projective spaces
\def\RP{\mathbb{RP}}
\def\CP{\mathbb{CP}}
\def\HP{\mathbb{HP}}


% Commen Cathegories
\DeclareMathOperator{\Top}{Top}
\DeclareMathOperator{\Ab}{Ab}
\DeclareMathOperator{\Cat}{Cat}
\DeclareMathOperator{\CAT}{CAT}
\DeclareMathOperator{\Mod}{Mod}
\DeclareMathOperator{\Ring}{Ring}
\DeclareMathOperator{\Group}{Group}
\DeclareMathOperator{\Sets}{Sets}


% Homological Algebra
\DeclareMathOperator{\Hom}{Hom}
\DeclareMathOperator{\Tor}{Tor}
\DeclareMathOperator{\Ext}{Ext}


% Common Lie Groups 
\DeclareMathOperator{\GL}{GL}
\DeclareMathOperator{\SU}{SU}
\DeclareMathOperator{\U}{U}
\DeclareMathOperator{\Sp}{Sp}

% Maths Operators
\DeclareMathOperator{\id}{id}
\DeclareMathOperator{\pr}{pr}


\DeclareMathOperator{\Ker}{Ker}
\DeclareMathOperator{\im}{Im}


% Standard setts.
\def \N {\mathbb{N}}
\def \Z {\mathbb{Z}}
\def \Q {\mathbb{Q}}
\def \R {\mathbb{R}}
\def \C {\mathbb{C}}
\def \H {\mathbb{H}}

\def \E {\mathbb{E}}
\def \Z {\mathbb{Z}}
\def \I {\mathbb{I}}
\def \J {\mathbb{J}}

% Vector calculus.
\newcommand{\dif}[3][]{
	\ensuremath{\frac{d^{#1} {#2}}{d {#3}^{#1}}}}
\newcommand{\pdif}[3][]{
	\ensuremath{\frac{\partial^{#1} {#2}}{\partial {#3}^{#1}}}}

% Vectors and matricies.
\newcommand{\mat}[1]{\begin{matrix} #1 \end{matrix}}
\newcommand{\pmat}[1]{\begin{pmatrix} #1 \end{pmatrix}}
\newcommand{\bmat}[1]{\begin{bmatrix} #1 \end{bmatrix}}

% Add space around the argument.
\newcommand{\qq}[1]{\quad#1\quad}
\newcommand{\q}[1]{\:\:#1\:\:}

% Implications
\newcommand{\la}{\ensuremath{\Longleftarrow}}
\newcommand{\ra}{\ensuremath{\Longrightarrow}}
\newcommand{\lra}{\ensuremath{\Longleftrightarrow}}

\newcommand{\pwf}[1]{\begin{cases} #1 \end{cases}}
\newcommand{\tif}{\text{if}}
\newcommand{\tand}{\text{and}}

% Shorthand
\newcommand{\vphi}{\varphi}
\newcommand{\veps}{\varepsilon}

\newcommand{\<}[1]{\langle #1 \rangle}

% Notation
\newcommand{\wddef}[1]{\underline{#1}}
\newcommand{\pref}[1]{(\ref{#1})}


% Maths Operators
\theoremstyle{plain}
\theoremstyle{definition}
\newtheorem{thrm}{Theorem}[section]
\newtheorem{prop}[thrm]{Proposition}
\newtheorem{corol}[thrm]{Corollary}
\newtheorem{lemma}[thrm]{Lemma}

\newtheorem{defn}[thrm]{Definition}
\newtheorem{exmp}[thrm]{Example}
\newtheorem{clame}[thrm]{Clame}


\theoremstyle{remark}
% \newtheorem{remark}[thrm]{\normalfont\large\textit Remark}
\newtheorem{remark}[thrm]{Remark}
\newtheorem{note}[thrm]{Note}


\newSimpleHeaderEnvironment{exercise}{Exercise }
\newSimpleHeaderEnvironment{solution}{Solution }

% \newColoredBoxEnvironment{exercise}{Exercise }{green!20}
% \newColoredBoxEnvironment{solution}{Solution }{blue!20}

\def\A{\mathbb{A}}
\DeclareMathOperator{\rad}{rad}
\DeclareMathOperator{\tr}{tr}

%%%%%%%% Content %%%%%%%%%%%%%%%%%%%%%%%%
\begin{document}
\mmaketitle

% \begin{exercise}[1]
% \end{exercise} 

\begin{solution}[1] %\vspace{-.8cm}
\begin{enumerate}
\item Define $f\in k[X_1,...,X_n]$ by $f(x) = \prod_{y \in Z} (x - y)$ for $x
\in \A^n(k)$. Then 
\[ V(f) = \{ x \in \A^n(k) \q: f(x) =0 \} = Z. \] Hence $Z$ is
closed wrt. the Zariski topology.
\item
The Zariski closed subsets of $\A^1(k)$ are precisely the finite subsets,
so the product topology on $\A^1(k) \times \A^1(k)$ is precisely the
finite subsets. However 
\[ \{ (x,-x) \q: x\in k \} = \{ (x,y) \q: x+y = 0 \}, \]
which is not finite (assuming $k$ is not finite), is closed in $\A^2(k)$. 
\end{enumerate}
\end{solution} 

\begin{solution}[2]
Let $f \in k[X_1,...,X_{n}]$ be non-constant. Then there exists $x_1 \ne x_2 \in
\A^n(k)$ such that $f(x) \ne f(y)$. 
Define $g \in k[X]$, by 
\[ g(x) = f(x x_1 + (1-x) x_2), \]
then $g$ is polynomial of one variable that is non-constant, since $g(0) =
f(x_2) \ne f(x_1) = g(1)$. 
So since $k$ is algebraically closed there exists $y \in k$ such that $g(y) =
0$ and thus 
\[ f(y x_1 - y x_2 + x_2) = 0, \]
hence $y (x_1 - x_2) + x_2$ is a zero of $f$.
\end{solution}

\begin{solution}[3]
\begin{enumerate}
\item By def
\item
Define $g(x_1, ..., x_{n+1}) = f(x_1, ..., x_n)x_{n+1} - 1$, then 
\begin{equation}
V(I(X)\cup\{g\}) = \im(D(f)) \subseteq \A^{n+1}(k),
\label{eq:*}
\end{equation}
where $\iota : I(X) \hookrightarrow k[x_1,..., x_{n+1}]$ in the obvious way.

\underline{proof of \pref{eq:*}} 
So the inclusion $\iota$ gives $V(I(X)) = X \times k \subseteq \A^{n+1}(k)$ ans clearly $g(x_1, ..., x_{n+1}) = 0$ implies $x_{n+1} = f^{-1}(x_1,...,x_n)$. So 
\begin{align*}
V(I(X) \cup \{g\}) &= V(I(X)) \cap V(g) \\
&= \{ (x_1,...,x_n, f^{-1}(x_1,...,x_n)) \in X \q: (x_1,...,x_n) \in X \text{ and } f(x_1, ..., x_n) \ne 0 \}, 
\end{align*}
which is clearly equal to $\im(D(f))$.

It remains to show that the coordinate ring of $\im(D(f))$ is given by the localisation $A[f^{-1}]$.

By \pref{eq:*} we have that the coordinate ring is given by 
\[ C = k[x_1,...,x_{n+1}] / (I(X)\cup \left<g\right>) \cong  A / \left<g\right> \]
Define $\psi: A[f^{-1}] \to A / \left<g\right>$ by sending $\frac{1}{f}$ to $x_{n+1}$ and $x_i$ to itself. This is well defined since $f x_{n+1} = 1$ in $A / \left<g\right>$ and also clearly a homeomorphism. $\psi$ is also clearly an isomorphism since it is bijective on the generators.
\end{enumerate}

\end{solution}


\begin{solution}[4]
We will prove TFAE
\begin{enumerate}
\item $A$ is nilpotent (ie. $A^n = 0$)
\item $A^2 = 0$, ie. $(a,b,c,d) \in V(I)$,
\item $\det A = 0$ and $\tr A = 0$, ie. $(a,b,c,d) \in V(J)$.
\end{enumerate}

\begin{enumerate}
\item[$(3)\implies(2)$]
By the Cayley-Hamilton theorem
\[ A^2 - \tr(A) A + \det(A) I_2 = 0. \]
By assumption $\tr(A) = 0$ and $\det(A) = 0$. So $A = 0$
\item[$(2)\implies(1)$]
By definition 
\item[$(1)\implies(3)$]
Consider an eigenvalue $\lambda$ of $A$, $A x = \lambda x$. By recursively
applying $A$ we have $A^n x = \lambda x$.
By assumtion $A^n = 0$ so $\lambda = 0$. This is true for any eigenvalue. 
The determinant is the product of the eigenvalues and the trace is the sum of
them, so $\det(A) = \tr(A) = $.
\end{enumerate}


We have $(a+d) \in J$ and $(a+d)$ has degree one, but all the polynomials in
$I$ have degree 2, so $(a+d) \not\in I$. Hence $J \ne I$.

By the nullstellensatz; $\rad(I) = I(V(I)) = I(V(J)) = \rad(J)$.

Furthermore we claim $\rad(J) = J$, so $\rad(I) = J$.

We have $J$ radical iff $k[a,b,c,d] / J$ is reduced, which is clear, since
both $ad-bc$ and $a+d$ are irreducible polynomials.
\end{solution}



\end{document}
