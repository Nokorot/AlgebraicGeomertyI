\documentclass[a4paper,11pt,english]{article}
\usepackage{.styles/basic}
\usepackage{.styles/envs}

%%%%%%% Title %%%%%%%%%%%%%%%%%%%%%%%%%%%
\title{\textbf{Algebraic Geometry} - Exercise Sheet 4}
\author{Tor Gjone \& Paul}

%%%%%%% Definitions %%%%%%%%%%%%%%%%%%%%%


% Algebraic Geometry stuff
\def\A{\mathbb{A}}
\DeclareMathOperator{\rad}{rad}
\DeclareMathOperator{\tr}{tr}

\DeclareMathOperator{\Sh}{Sh}
\DeclareMathOperator{\op}{^{op}}
\DeclareMathOperator{\Id}{Id}
\DeclareMathOperator{\Ouv}{Ouv}

\DeclareMathOperator{\Ann}{Ann}
\DeclareMathOperator{\Supp}{Supp}
\DeclareMathOperator{\res}{res}
\newcommand{\B}{\mathcal{B}}
\newcommand{\F}{\mathcal{F}}
\newcommand{\G}{\mathcal{G}}


% Direct limit and colimit
\setlength{\parindent}{0cm}
\setlength{\parskip}{3pt}

\makeatletter
\newcommand{\lim@}[2]{%
  \vtop{\m@th\ialign{##\cr
    \hfil$#1\operator@font lim$\hfil\cr
    \noalign{\nointerlineskip\kern1.5\ex@}#2\cr
    \noalign{\nointerlineskip\kern-\ex@}\cr}}%
}
\newcommand{\colim}{%
  \mathop{\mathpalette\varlim@{\rightarrowfill@\scriptscriptstyle}}\nmlimits@
}
\newcommand{\invlim}{%
  \mathop{\mathpalette\varlim@{\leftarrowfill@\scriptscriptstyle}}\nmlimits@
}
\makeatother

% Tor's stuff
\renewcommand{\i}{^{-1}}

\newcommand{\til}{\tilde}


\newcommand{\sm}{\setminus}
\newcommand{\es}{\emptyset}
\newcommand{\se}{\subseteq}

\newcommand{\oc}[1]{\overset{\circ}{#1}}
\newcommand{\into}{\hookrightarrow}
\newcommand{\nb}{\nabla}
\renewcommand{\hat}{\widehat}

\usepackage{scalerel,stackengine}
\stackMath
\renewcommand\widehat[1]{%
\savestack{\tmpbox}{\stretchto{%
  \scaleto{%
    \scalerel*[\widthof{\ensuremath{#1}}]{\kern.1pt\mathchar"0362\kern.1pt}%
    {\rule{0ex}{\textheight}}%WIDTH-LIMITED CIRCUMFLEX
  }{\textheight}% 
}{2.4ex}}%
\stackon[-6.9pt]{#1}{\tmpbox}%
}




\newcommand{\diagSquare}[8]{
\begin{tikzpicture}[node distance=2cm, auto]
\node (a)              { $ #5 $ };
\node (b) [right of=a] { $ #6 $ };
\node (c) [below of=a] { $ #7 $ };
\node (d) [right of=c] { $ #8 $ };
\draw[-to] (a) to node { $ #1 $ } (b);
\draw[-to] (a) to node { $ #2 $ } (c);
\draw[-to] (b) to node { $ #3 $ } (d);
\draw[-to] (c) to node { $ #4 $ } (d);
\end{tikzpicture}
}


\newcommand{\congto}{\xrightarrow{\cong}}
\newcommand{\incto}[1][]{ \xhookrightarrow{#1} }
\newcommand{\xto}[1]{\xrightarrow{#1}}


% Projective spaces
\def\RP{\mathbb{RP}}
\def\CP{\mathbb{CP}}
\def\HP{\mathbb{HP}}


% Commen Cathegories
\DeclareMathOperator{\Top}{Top}
\DeclareMathOperator{\Ab}{Ab}
\DeclareMathOperator{\Cat}{Cat}
\DeclareMathOperator{\CAT}{CAT}
\DeclareMathOperator{\Mod}{Mod}
\DeclareMathOperator{\Ring}{Ring}
\DeclareMathOperator{\Group}{Group}
\DeclareMathOperator{\Sets}{Sets}


% Homological Algebra
\DeclareMathOperator{\Hom}{Hom}
\DeclareMathOperator{\Tor}{Tor}
\DeclareMathOperator{\Ext}{Ext}


% Common Lie Groups 
\DeclareMathOperator{\GL}{GL}
\DeclareMathOperator{\SU}{SU}
\DeclareMathOperator{\U}{U}
\DeclareMathOperator{\Sp}{Sp}

% Maths Operators
\DeclareMathOperator{\id}{id}
\DeclareMathOperator{\pr}{pr}


\DeclareMathOperator{\Ker}{Ker}
\DeclareMathOperator{\im}{Im}


% Standard setts.
\def \N {\mathbb{N}}
\def \Z {\mathbb{Z}}
\def \Q {\mathbb{Q}}
\def \R {\mathbb{R}}
\def \C {\mathbb{C}}
\def \H {\mathbb{H}}

\def \E {\mathbb{E}}
\def \Z {\mathbb{Z}}
\def \I {\mathbb{I}}
\def \J {\mathbb{J}}

% Vector calculus.
\newcommand{\dif}[3][]{
	\ensuremath{\frac{d^{#1} {#2}}{d {#3}^{#1}}}}
\newcommand{\pdif}[3][]{
	\ensuremath{\frac{\partial^{#1} {#2}}{\partial {#3}^{#1}}}}

% Vectors and matricies.
\newcommand{\mat}[1]{\begin{matrix} #1 \end{matrix}}
\newcommand{\pmat}[1]{\begin{pmatrix} #1 \end{pmatrix}}
\newcommand{\bmat}[1]{\begin{bmatrix} #1 \end{bmatrix}}

% Add space around the argument.
\newcommand{\qq}[1]{\quad#1\quad}
\newcommand{\q}[1]{\:\:#1\:\:}

% Implications
\newcommand{\la}{\ensuremath{\Longleftarrow}}
\newcommand{\ra}{\ensuremath{\Longrightarrow}}
\newcommand{\lra}{\ensuremath{\Longleftrightarrow}}

\newcommand{\pwf}[1]{\begin{cases} #1 \end{cases}}
\newcommand{\tif}{\text{if}}
\newcommand{\tand}{\text{and}}

% Shorthand
\newcommand{\vphi}{\varphi}
\newcommand{\veps}{\varepsilon}

\newcommand{\<}[1]{\langle #1 \rangle}

% Notation
\newcommand{\wddef}[1]{\underline{#1}}
\newcommand{\pref}[1]{(\ref{#1})}


% Maths Operators
\theoremstyle{plain}
\theoremstyle{definition}
\newtheorem{thrm}{Theorem}[section]
\newtheorem{prop}[thrm]{Proposition}
\newtheorem{corol}[thrm]{Corollary}
\newtheorem{lemma}[thrm]{Lemma}

\newtheorem{defn}[thrm]{Definition}
\newtheorem{exmp}[thrm]{Example}
\newtheorem{clame}[thrm]{Clame}


\theoremstyle{remark}
% \newtheorem{remark}[thrm]{\normalfont\large\textit Remark}
\newtheorem{remark}[thrm]{Remark}
\newtheorem{note}[thrm]{Note}


\newSimpleHeaderEnvironment{exercise}{Exercise }

\DeclareMathOperator{\OC}{OC}
\DeclareMathOperator{\Idem}{Idem}

%%%%%%%% Content %%%%%%%%%%%%%%%%%%%%%%%%
\begin{document}
\mmaketitle

\renewcommand{\O}{\mathcal{O}}

\begin{exercise}[2]
\begin{enumerate}
\item
$\{ V \subset U \} \cap \{ V \subset X \sm U \}$ defines a cover of $X$, so by 
the definition of a sheaf there exists a unique $e_U \in \Gamma(X,\O_X)$ such
that 
\[ e_U |_V = 1 \:\tif\: V \subset U \q{\tand} e_U|_V = 0 \:\tif\: V \subset X
\sm U. \]
Note that $U$ is open and $X\sm U$ is open, so $e_U = 1$ and $e_U|_{X\sm U} =
0$. 

If $V \se U$ then, by the functorialety of the restriction map $(e_U e_U) |_V =
1 \cdot 1 = 1$ and similarly of $V \se X \sm U$, $(e_U e_U)|_V = 0$.
So by the uniqueness, from above, $e_U e_U = e_U$ and thus $e_U$ is idempotent.

If $V \subset X$ is another clopen set. ie. 
\[ V \cap (X\sm U) \es \q\lor U \cap (X \sm V) \ne \es. \]
By symmetry we may wlog. assume 
\[ W = V \cap (X \sm U) \ne \es. \]

Since $V$ and $U$ are both clopen, $W$ most also be clopen and we have 
$W \se V$ and $W \se X\sm U$. So $e_V|_W = 1$  but $e_U|_W = 0$. So 
\[ \Psi : \OC(X) \to \Idem(\Gamma(X,\O_X));\: U \mapsto e_U \]
is an injection.

Let $e \in \Idem(\Gamma(X,\O))$. Then for $x\in X$, $e_x=1$ or $e_x=0$, since 
$e_x$ is idempotent in $(\O_X)_x$ ( $e_x = (e\cdot e)_x = e_x \cdot e_x$.)

Since $(\O_X)_x$ is a local ring $e_x$ is a unit or $1-e_x$ is a unit.
We have 
\begin{align}
ae_x = 1 \q\ra ae_x e_x = e_x \q\ra e_x = ae_x = 1, \\
%
a (1 - e_x) = 1 \q\ra a(e_x - e_x^2) = e_x \q\ra e_x = a(0) = 0.
\end{align}
So $e_x = 1$ or $e_x = 0$. 

Let $U = \{ x \in X \q| e_x = 1\}$. The we want to show that $U$ is clopen. 
Let $x\in U$, then by the definition of the stalk, there exists $V \se X$ such
that $x \in V$ and $e|_V = 0$. So $V$ is a neighbourhood of $x$ in $U$ and thus
$U$ is open. Similarly let $x \in X\sm U$, then $e_x = 1$, so there exists a
neighbourhood $V \se X$ of $x$ such $e|_V = 1$. So $V \subset X\sm U$. Hence $U$
is closed and thus clopen. 
Clearly $e = e_U$.

\item 
If $V \se U \cap U'$, $(e_U e_{U'})|_V = 1$, by functorialety of the restiction,
like above. 
If $V \se X \sm (U \cap U') = (X \sm U) \cup (X \sm U')$, 
let $V_1 = (X \sm U) \cap V$ and $V_2 = (X \sm U') \cap V$. Then, 
$e_U|_{V_1} = e_{U'}|_{V_2} = 0$, so for $i=1,2$,
\[ (e_U e_{U'})|_{V_i} = e_U|_{V_i} e_{U'}|_{V_i} = 0. \]
Since $\{ V_1, V_2 \}$ defines a cover of $V$, we can conclude that $(e_U
e_{U'})|_V = 0$. Hence $e_U e_{U'} = e_{U\cap U'}$, by the uniqueness from part
one.

\item 
If $X$ is connected there are no proper non-empty clopen subsets of $X$, so by
part one there are no idempotents $e\ne 0,1$. 

Conversely if $X$ is not connected, there exists a proper, non-empty, clopen
$U \se X$ and $e_U \ne 0,1$ defines an idempotent.

Suppose $\Gamma(X, \O_X) \cong R_1 \times R_2$ (where $R_1$ and $R_2$ are
non-zero rings), then $(1,0) \in \R_1 \times R_2$ defines an idempotent $\ne
0,1$.

Conversely if $e \ne 0,1$ is an idempotent in $R = \Gamma(X,\O_X)$, then 
$Re$ and $R(1-e)$ defines two non zero ring such that $R \cong Re \times
R(1-e)$. Since $e$ is non-zero $e \in Re$ is non-zero. Similarly since $e \ne
1$, $(1-e) \in R(1-e)$ is non-zero.

Since $Re$ is an ideal, it is enough to show that there is a unit element to
show that $Re$ is a ring. $e \in Re$ is a unit element, since for all $a \in R$ 
$(ae)e = ae$. So $Re$ is a ring. The same argument works for $R(1-e)$, since
$(1-e)(1-e) = 1 - e - e + e = 1-e$ and thus $(1-e)$ is an idempotent.

To show that $R \cong Re \times R(1-e)$. We define $\psi: R \to Re \times
R(1-e)$ by $r \mapsto (re, r(1-e))$ and $\phi: Re \times R(1-e)\to R$ by 
$(a,b) \mapsto a+b$. Clearly this defines ring homomorphisms, and in fact they
defines mutual inverses.

Let $r \in R$, then $(\phi \circ \psi)(r) = re + r(1-e) = r$. Conversely let
$(ae,b(1-e)) \in Re\times R(1-e)$, then 
\begin{align*}
(\psi \circ \phi)(ae,a(1-e)) &= \left((ae + b(1-e))e, (ae+b(1-e)) (1-e)\right)
\\
&= \left( ae + b(e-e), a(e-e) + b(1-e) \right ) \\
&= \left(ae,b(1-e)\right) 
\end{align*}
\end{enumerate}
\end{exercise}

\begin{exercise}[3]
\begin{enumerate}
\item 
Let $p \in Z$ and $(a,V) \in \F_p$, that is $V \se Z$ open and $a \in \F(V)$. 
By the definition of subspace topology there exists $\til V\se X$, such that 
$V = \til V|_Z = i\i(\til V)$. So $a \in \F(V) = \F(i\i(\til V))$ and thus
$(a, \til V) \in (i_*\F)_p$. 

Conversely, let $(a,V) \in (i_*\F)_p$, then $a \in \F(i\i(V))$. So
$(a,i\i(V)) \in \F_p$. This construction is clearly mutually inverse.

If $p \ne Z$, then $X\sm U$ is a neighbourhood of $p$ and $\F(i\i(X \sm U)) =
\F(\es) = 0$. So $(i_*\F)_p = 0$.

\item 
Since the sheafification does not change the stalks, the stalk $(j_!\F)_p$ is
given by pairs $(a,V)$, for $p \in V \se U$ and $a \in \F(V)$, and $(0, V)$, for
$V \not\se U$. Also, since $U$ is open, $V \se U$ is open in $U$ iff $V$ is open
in $X$.

Hence clearly $(j_!\F)_p = p$ if $p \in U$ and $(j_!\F)_p = 0$ if $p \ne U$.

\item 
I suffices to show the exact sequence on stalks.

Consider $p \in U$, then 
\begin{align*}
(j_!(j\i\F))_p &= (j\i\F)_p, && \text{by part 2} \\
&= (j^+\F)_p, && \text{since the stalk is preserved by sheafification} \\
&= \F_p, &&\text{since } j \text{ is an open embedding. And}  \\
%
(i_*(i\i\F))_p &= 0, && \text{by part 1.}
\end{align*}
If $p \in Z$, then
\begin{align*}
(j_!(j\i\F))_p &= 0, && \text{by part 2, and} \\
(i_*(i\i\F))_p &= (i\i\F)_p, && \text{by part 1.} \\
&= (i^+\F)_p, \\
&= \colim_{p \in V \in \O_X} \F(V) && \text{colimit of colimit and since }
j(p) = p, \\
&= \F_p && \text{by definition of stalk.}
\end{align*}

So on the stalks, the sequence has the form 
\[ \begin{tikzcd}[row sep=.5em]
0 \arrow[r]& \F_p \arrow[r]& \F_p \arrow[r]& 0 \arrow[r]& 0 \\
0 \arrow[r]& 0 \arrow[r]& \F_p \arrow[r]& \F_p \arrow[r]& 0
\end{tikzcd} \]
where in the first line $p \in U$ and in the second $p \in Z$.

Since the maps are defined by the inclusions it is clear that these sequences are
exact. ie. the map $\F_p \to \F_p$, is an isomorphism.

\item % 4
This question does not make sense.

We have
\begin{align*}
(j_*(j\i\F))_0 &= (j\i\F)_0 && \text{by part 1} \\
&= \colim_{0 \in V \in \O_\C} \F(V) = \F_0
\end{align*}
But $(j_*(j\i\F))_x = 0$, for any point $x \ne 0$. So any map 
$\F \to j_*(j\i\F)$ cannot be injective.


On the other hand $(j_!(j\i\F))_0 = 0$, by part 2, and the map 
$j_!(j\i\F) \to \F$, for part 3, is injective, but cannot be surjective,
since $\F_x \ne 0$.


\end{enumerate}

\end{exercise}



\end{document}
